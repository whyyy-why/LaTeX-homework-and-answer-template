\documentclass{article}
\usepackage[UTF8]{ctex}   %中文
\ctexset{
	proofname = \heiti{证明}
}
\usepackage{moreenum}
\usepackage{mathtools}
\usepackage{url}
\usepackage{bm}
\usepackage{enumitem}
\usepackage{graphicx}
\usepackage{subcaption}
\usepackage{amsmath,amsthm}
\usepackage{booktabs}
\usepackage{fancyhdr} % 页眉页脚
\usepackage{lastpage} % 获得总页数

\raggedright % 防止右边界越界

% 首行缩进设置
\usepackage{indentfirst}
\usepackage{amssymb}
\setlength{\parindent}{0em}  %不缩进

%页眉页脚
\pagestyle{fancy}
\lhead{}                     %页眉左   
\chead{\bfseries 数理方程B}  %中                      
\rhead{\small\leftmark}     %右                           
\cfoot{\thepage\ of \pageref{LastPage}}   %当前页面of总页码
\rfoot{}                                             \lfoot{}                                             
\renewcommand{\headrulewidth}{0.5pt}%改为0pt即可去掉页眉下面的横线
%\setlength{\skip\footins}{0.5cm}    %脚注与正文的距离           
%\renewcommand{\footnotesize}{}      %设置脚注字体大小           
\renewcommand{\footrulewidth}{0pt}  %脚注线的宽度  


% 封面
\title{\textbf{\underline{数理方程 作业1}}} %标题
\date{}              %日期 为空则不写日期 否则自动
\author{}            %作者



% 正文

\begin{document}
	\large
	\maketitle
	%题目
	\paragraph{5.}{\scriptsize \large 证明:\[u = f(x,y)\]满足方程 }           	
	 \large  \[ xu_x-yu_y=0 \]
	 
	%证明
    \begin{proof}
 	\begin{equation}
	\large
	\begin{aligned}
	 u_x&=yf'(xy) \\
	   u_y&=xf'(xy) \\
	 \therefore xu_x &-yu_y=0 \\
	 \end{aligned}
	\end{equation}
   \end{proof}
	
  %下一题

	\paragraph{6.}{\large 设u=u(x,y,z),求下列方程的通解:}\\

	{\large \  (1) $ u_y + a(x,y)u = 0 $;}\\
	{\large \ (2) $ u_{xy} + u_y  = 0 $;  }\\
% 	{\large \ (3)$  u_{tt}= a^{2}u_{xx} + 3x^{2}$  (设$u=u(x,t)  $).	}\\
	~\\

解:\large (1)由方程,
\begin{flushleft}
	$\begin{aligned}
	\frac{{\partial u}}{{\partial y}} &=  - au\\
	\frac{1}{u}\frac{{\partial u}}{{\partial y}} &=  - a\\
    \frac{\partial }{{\partial y}}\ln |u| &=  - a\\
	\ln|u|& = \int 
	{\frac{\partial }{{\partial {\rm{y}}}}\ln udy + \varphi (x) =  - \int {a(x,y)dy} }  + \varphi (x)\\
	&u =  \pm f(x)\exp \{  - \int {a(x,y)dy} \} .其中f(x) = {e^{\varphi (x)}}
	\end{aligned}$		
	\end{flushleft}


~\\
(2)\setlength{\parindent}{2em}
 \large
  
   令 $ p(x,y) = {u_y}(x,y)$,则	\\
	
	$\begin{aligned}
	 &{p_x} + p = 0\\
     &\therefore {(\ln|p|)_x} =  - 1\\
     &\therefore p = \varphi (y){e^{ - x}}\\
     &\therefore {u_y} = \varphi (y){e^{ - x}}\\
     &\therefore	 u = f(y){e^{ - x}} + g(x) \\
     \end{aligned}$

	\paragraph{7.}
	{\large 设有一根具有绝热的侧表面的均匀细杆, 它的初始温度为$ \varphi(x) $, 两端满足下列边界条件之一:\\
		(1)	一端 (x = 0) 绝热, 另一端 (x = l) 保持常温$ u_0 $; \\
		(2)	两端分别有恒定的热流密度 $ q_1  $ 及$  q_2  $进入;\\
		(3)	一端 (x = 0) 温度为$  \mu(t) $, 另一端 (x = l) 与温度为 $ \theta(t)  $的介质有热交换,\\
		试分别写出上述三种热过程的定解问题.\\
	}
~\\
     \setlength{\parindent}{0em}
     解:\large
     泛定方程为:$ 	{u_t} = {a^2}{u_{xx}} $\\
     \setlength{\parindent}{2em}
     初始条件为:$ 	u(0,x) = \varphi (x) $\\
     边界条件分别为:\\

	$	\begin{aligned}
     (1)&{u_x}(t,0) = 0,{u_x}(t,l) = {u_0}\\
     (2)&{u_x}(t,0) =  - \frac{{{q_1}}}{k},{u_x}(t,l) = \frac{{{q_2}}}{k}\\
     (3)&u(t,0) = \mu (t),(k{u_x} + hu){|_{x = l}} = h(l)\theta (t)
	\end{aligned}$
~\\
   \paragraph{8.}
   {\large 一根长为 l 而两端 (x = 0 和 x = l) 固定的弦, 用手把它的中点朝横向拨开距离 h, 然后放手任其自由振动, 试写出此弦振动的定解问题\\	
   }
~\\
   \setlength{\parindent}{0em}
	解:
    \large
    \[
     \left\{ \begin{array}{l}
     {u_{tt}} = {a^2}{u_{xx}}\\
      u(0,x) = \left\{ {\begin{array}{*{20}{l}}
  	        {\frac{{2h}}{l}x{{\qquad\quad}\text{if}}\ 0 \le x \le \frac{l}{2},}\\
	        {\frac{{2h}}{l}(l - x) \ \ \ \text{if}\ \frac{l}{2} < x \le l,}
         	\end{array}} \right.\\
      {u_t}(0,x) = 0,u(t,0) = u(t,l) = 0
      \end{array}\right.
     \]

\end{document}
